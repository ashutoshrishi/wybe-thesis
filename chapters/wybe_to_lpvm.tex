\chapter{Transforming Wybe to LPVM}
\label{chap:wybe_to_lpvm}

\begin{figure}
  \centering
  \begin{tabular}{r c l}

    \( func\ factorial(n:int):int \) & \(\rightarrow \) & 
                                                          \( proc\ factorial(n:wybe.int, ?\verb+$+:wybe.int) \) \\
    \( ?c = bar(a, b) \) & \(\rightarrow\) & \( bar(a, b, ?c) \) \\
    \( ?y = f(g(x)) \) & \(\rightarrow\) & \( g(x, ?temp)\ f(temp, ?y) \) \\


  \end{tabular}
  \\
  \caption{Normalisation of Wybe functions to Procedures.}
  \label{fig:wybe_convert_to_proc}
\end{figure}



\begin{figure}
  \centering
  \begin{tabular}{r c l}
    \( ProcDef \)     & \( \rightarrow \) & \( ProcImpln* \)   \\
    \( ProcImpln \)   & \( \rightarrow \) & \( ProcDefSrc \)   \\
                      & \( \rvert \)        & \( ProcDefPrim \)  \\
    \( ProcDefSrc \)  & \( \rightarrow \) & \( Stmt* \)        \\
    \( ProcDefPrim \) & \( \rightarrow \) & \( Proto\ Body \)  \\
  \end{tabular}
  \caption{The Procedure Implementation Algebraic Data Type}
  \label{fig:proc_impln}
\end{figure}



The Wybe syntax tree is slowly transformed to the LPVM IR structure. In this
process it undergoes \textit{flattening}, \textit{type checking},
\textit{un-branching}, and a final clause generation pass to obtain a structure
similar to Figure~\ref{fig:lpvm_data_type}. Once the LPVM structure is
obtained, we can run optimisation passes over it.

The type which stores the implementation of a Wybe procedure, in source and
LPVM form, is shown in Figure~\ref{fig:proc_impln}. A procedure definition
\textit{ProcDef} will also contain other information about the callers, visible
types, and more. A procedure can have multiple implementation, each
implementation corresponding to a different \textit{Clause} of the
procedure. Initially a \textit{ProcImpln} will be composed from the constructor
\textit{ProcDefSrc}, indicating source language form. On transformation to
LPVM, the same type will have the constructor \textit{ProcDefPrim}. The
\textit{Proto} and \textit{Body} are similar to the constructors in
Figure~\ref{fig:lpvm_data_type}.

The compiler implementation keeps the pipeline modular. While the module
implementations are stored in a \textit{List} data structure, it is possible to
generate a module dependency graph given a module. The implementation for any
given module name can by pulled from or place in the module list store. The
alternative approach of always maintaining a dependency \textit{Map} would have
made jumping into and out of module implementations more costly since these
operations are done multiple times during a pass. This also makes loading and
removing module implementations in the pipeline very easy, assisting the
incremental features later.



\subsection{Flattening Pass}

During compilation everything is converted to a procedure quite early
in the pipeline. The functions and expressions are normalised to look like a
procedure definition along with the flattening step by the compiler. The output
of the function is simply added as a out flowing parameter in its procedure
form. Expressions are dealt with in a similar way.  Some common conversions are
shown in Figure~\ref{fig:wybe_convert_to_proc}.

Since LPVM primitives are in the form of procedure calls, all normalised Wybe
statements are gradually reduced to procedure calls too. These primitive
procedure calls can be calls to other procedures in the module or imported
modules (fully qualified procedure names), or be foreign calls. Foreign calls
reference procedures or instructions which have to be addressed later by
linking in some library which provides it. For example, the wybe standard
library defines \textit{println} whose body statements are foreign calls to C's
\textit{printf}. A shared C library will be linked with the standard library to
resolve these calls to access system IO later. To Wybe and LPVM the only
difference between a local and a foreign procedure call is that the local calls
can be in-lined since their definitions will have a LPVM form in another wybe
module. Otherwise it is just another \textit{Prim} (primitive) in a LPVM clause
body.

\subsection{Type Checking Pass}

Wybe is statically typed, so having a type checking pass is essential. Every
variable name in the AST will be annotated with an inferred type. This pass
connects type names to the modules that provide the definition for that
type. This is required as even standard types like \textit{int} can be provided
by a non standard library just as easily. Polymorphic calls are resolved to the
actual definitions here.

Type definitions include functions and procedures which work with the defined
type. For example, the equality function \textit{'='}, can be defined in a type
module for \textit{int} and the type module for \textit{string}. The type
checker will choose one depending on the context. A statement comparing two
\textit{int} (inferred) variables the call \textit{proc call =(a, b, ?c)}, will
be converted to \textit{proc call wybe.int.=(a:wybe.int, b:wybe.int,
  ?c:wybe.int)}. By the end of a successful type checking, every flattened
procedure call and variable types names will have an annotation of the fully
qualified module that defines it.


\subsection{Un-branching Pass}

The un-branching pass is where all conditional branches and loops are replaced
with procedure calls and recursion respectively. This is the structure defined
by LPVM. At this stage, a flattened Wybe procedure may create one or more
generated procedures to act as branching blocks, as described in the chapter on
LPVM.


\subsection{Clause Generation Pass}

By this pass the LPVM structure is more or less completed. The Wybe source
semantics have been encapsulated in the LPVM IR, at least what is needed. The
implementations for all the flattened clauses of the procedures will have the
constructor \textit{ProcDefPrim} applied, with the LPVM prototype and body
primitives put in its place. Further optimisation passes will work on this
structure until code generation.


\subsection{Expansion Pass}

An early optimisation that makes use of the LPVM structure is the in-lining of
calls. This pass takes place on the clausal or LPVM form (\textit{ProcDefPrim}
constructor in the implementation). LPVM makes heavy use of procedure calls
since every form of branching is essentially replaced with procedure
calls. In-lining bodies of \textit{callee} procedures removes most of these
procedure call overheads. The decision to inline or not is based on a cost vs
benefit analysis. This analysis takes into consideration the number of callers
to that procedure, it's arguments, and body size. 

In-lining benefits are only decidable if we have the LPVM clause form for that
procedure. The structure of LPVM procedures is much more flat and limited as
compared to Wybe source code, making copying procedure bodies easier, and the
decision to do so more fine grained. This is one of the benefits of having LPVM
forms of all modules involved in compilation in the compiler pipeline, and will
be reflected in the incremental features later.

As an example, most of Wybe standard library types like \textit{int}
,\textit{bool}, \textit{float}, have simple primitive procedures for binary
operations. The implementation bodies for these procedures are just a single
foreign instruction call to LLVM instructions. These will definitely be
in-lined and in code generation phase will turn into a single LLVM instruction.


%%% Local Variables:
%%% mode: latex
%%% TeX-master: "../thesis"
%%% End:
