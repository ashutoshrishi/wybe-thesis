\chapter{Transforming Wybe to LPVM}

\begin{figure}
  \centering
  \begin{tabular}{r c l}

    \( func\ factorial(n:int):int \) & \(\rightarrow \) & 
                                                          \( proc\ factorial(n:wybe.int, ?\verb+$+:wybe.int) \) \\
    \( ?c = bar(a, b) \) & \(\rightarrow\) & \( bar(a, b, ?c) \) \\
    \( ?y = f(g(x)) \) & \(\rightarrow\) & \( g(x, ?temp)\ f(temp, ?y) \) \\


  \end{tabular}
  \\
  \caption{Normalisation of Wybe functions to Procedures.}
  \label{fig:wybe_convert_to_proc}
\end{figure}



\begin{figure}
  \centering
  \begin{tabular}{r c l}
    \( ProcDef \)     & \( \rightarrow \) & \( ProcImpln* \)   \\
    \( ProcImpln \)   & \( \rightarrow \) & \( ProcDefSrc \)   \\
                      & \( \rvert \)        & \( ProcDefPrim \)  \\
    \( ProcDefSrc \)  & \( \rightarrow \) & \( Stmt* \)        \\
    \( ProcDefPrim \) & \( \rightarrow \) & \( Proto\ Body \)  \\
  \end{tabular}
  \caption{The Procedure Implementation Algebraic Data Type}
  \label{fig:proc_impln}
\end{figure}



The Wybe syntax tree is slowly transformed to the LPVM IR structure. In this
process it undergoes \textit{flattening}, \textit{type checking},
\textit{un-branching}, and a final clause generation pass to obtain a structure
similar to Figure~\ref{fig:lpvm_data_type}. The type which stores the
implementation of a Wybe procedure, in source and LPVM form, is shown in
Figure~\ref{fig:proc_impln}. A procedure definition \textit{ProcDef} will also
contain other information about the callers, visible types, and more. A
procedure can have multiple implementation, each implementation corresponding
to a different \textit{Clause} of the procedure. Initially a \textit{ProcImpln}
will be composed from the constructor \textit{ProcDefSrc}, indicating source
language form. On transformation to LPVM, the same type will have the
constructor \textit{ProcDefPrim}. The \textit{Proto} and \textit{Body} are
similar to the constructors in Figure~\ref{fig:lpvm_data_type}.

The compiler implementation keeps the pipeline modular. While the module
implementations are stored in a \textit{List} data structure, it is possible to
generate a module dependency graph given a module. The implementation for any
given module name can by pulled from or place in the module list store. The
alternative approach of always maintaining a dependency \textit{Map} would have
made jumping into and out of module implementations more costly since these
operations are done multiple times during a pass. This also makes loading and
removing module implementations in the pipeline very easy, assisting the
incremental features later.

\subsection{Flattening Pass}

During compilation everything is converted to a procedure quite early
in the pipeline. The functions and expressions are normalised to look like a
procedure definition along with the flattening step by the compiler. The output
of the function is simply added as a out flowing parameter in its procedure
form. Expressions are dealt with in a similar way.  Some common conversions are
shown in Figure~\ref{fig:wybe_convert_to_proc}.

Since LPVM primitives are in the form of procedure calls, all normalised Wybe
statements are gradually reduced to procedure calls too. These primitive
procedure calls can be calls to other procedures in the module or imported
modules (fully qualified procedure names), or be foreign calls. Foreign calls
reference procedures or instructions which have to be addressed later by
linking in some library which provides it. For example, the wybe standard
library defines \textit{println} whose body statements are foreign calls to C's
\textit{printf}. A shared C library will be linked with the standard library to
resolve these calls to access system IO later. To Wybe and LPVM the only
difference between a local and a foreign procedure call is that the local calls
can be in-lined since their definitions will have a LPVM form in another wybe
module. Otherwise it is just another \textit{Prim} (primitive) in a LPVM clause
body.

\subsection{Type Checking Pass}

Wybe is statically typed, so having a type checking pass is essential. Every
variable name in the AST will be annotated with an inferred type. This pass
connects type names to the modules that provide the definition for that
type. This is required as even standard types like \textit{int} can be provided
by a non standard library just as easily. Polymorphic calls are resolved to the
actual definitions here.

Type definitions include functions and procedures which work with the defined
type. For example, the equality function \textit{'='}, can be defined in a type
module for \textit{int} and the type module for \textit{string}. The type
checker will choose one depending on the context. A statement comparing two
\textit{int} (inferred) variables the call \textit{proc call =(a, b, ?c)}, will
be converted to \textit{proc call wybe.int.=(a:wybe.int, b:wybe.int,
  ?c:wybe.int)}. By the end of a successful type checking, every flattened
procedure call and variable types names will have an annotation of the fully
qualified module that defines it.


\subsection{Un-branching Pass}

The un-branching pass is where all conditional branches and loops are replaced
with procedure calls and recursion respectively. This is the structure defined
by LPVM. At this stage, a flattened Wybe procedure may create one or more
generated procedures to act as branching blocks, as described in the chapter on
LPVM.



% talk about inlining 



