\chapter{Transforming Wybe to LPVM}
\label{chap:wybe_to_lpvm}

\begin{figure}
  \centering
  \begin{tabular}{r c l}

    \( func\ factorial(n:int):int \) & \(\rightarrow \) & 
                                                          \( proc\ factorial(n:int, ?\verb+$+:int) \) \\
    \( ?c = bar(a, b) \) & \(\rightarrow\) & \( bar(a, b, ?c) \) \\
    \( ?y = f(g(x)) \) & \(\rightarrow\) & \( g(x, ?temp)\ \wedge f(temp, ?y) \) \\
    \( proc\ greet(s:string)\ use\ !io \) & \(\rightarrow\) &
                                                           \( proc\
                                                           greet(s:string,
                                                           \#0:io, ?\#1:io) \)
    \\
    \(?a = b + (c * d)\) & \(\rightarrow\) & \(mul(c,d,?t) \wedge add(b,t,?a) \) \\


  \end{tabular}
  \\
  \caption{Normalisation of Wybe semantic structures to Procedures}
  \label{fig:wybe_convert_to_proc}
\end{figure}



\begin{figure}
  \centering
  \begin{tabular}{r c l}
    \( ProcDef \)     & \( \rightarrow \) & \( ProcImpln* \)   \\
    \( ProcImpln \)   & \( \rightarrow \) & \( ProcDefSrc \)   \\
                      & \( \rvert \)        & \( ProcDefPrim \)  \\
    \( ProcDefSrc \)  & \( \rightarrow \) & \( Stmt* \)        \\
    \( ProcDefPrim \) & \( \rightarrow \) & \( Proto\ Body \)  \\
  \end{tabular}
  \caption{The Procedure Implementation Algebraic Data Type}
  \label{fig:proc_impln}
\end{figure}



The Wybe syntax tree is systematically transformed into the LPVM IR data
structure. In this process it undergoes \textit{flattening}, \textit{type
  checking}, \textit{un-branching}, and a final \textit{clause generation} to
obtain a structure similar to Figure~\ref{fig:lpvm_data_type}. Once the LPVM
structure is obtained, we can run more optimisation passes over it. One such
optimisation on a complete LPVM form is the \textit{Expansion} pass. This pass
explores costs and benefits of in-lining procedure bodies for calls, marking
calls which should be in-lined. In-lining at the LPVM level provides a lot of
benefits which are apparent in the incremental features of the compiler
discussed in Chapter~\ref{chap:build_system}.

The type which stores the implementation of a Wybe procedure, in source and
LPVM form, is shown in Figure~\ref{fig:proc_impln}. The complete procedure
definition \textit{ProcDef} will also contain other information about the
callers, visible types, and more. A procedure can have multiple
implementations, each of which corresponds to a different \textit{Clause} of
the procedure. Initially a \textit{ProcImpln} will be built from the
constructor \textit{ProcDefSrc}, indicating source language form. On completing
transformation to LPVM, the same type will have the constructor
\textit{ProcDefPrim}. The \textit{Proto} and \textit{Body} are similar to the
constructors in Figure~\ref{fig:lpvm_data_type}.

The compiler implementation keeps the pipeline modular. While the module
implementations are stored in a flat \textit{List} data structure, it is
possible to generate a module dependency graph (transitive closure of
dependencies) given a module name. The implementation for any given module name
can be obtained easily as long as it exists in the above list. The module
currently under compilation will be at the head of this list. This also makes
loading implementations from an external source and removing module
implementations from the pipeline very easy, assisting the incremental features
discussed in Chapter~\ref{chap:build_system}.

Since LPVM procedure signatures should also present a pure interface like Wybe,
Wybe \textit{resources} are passed as a new extra parameter to the
procedure. LPVM has no primitive syntax for Wybe \textit{resources}. Hence, if
a Wybe procedure uses the \textit{io} resource, its corresponding LPVM clause
will have an extra input and output parameter of the \textit{io} type. An
example of this is shown in Figure~\ref{fig:wybe_convert_to_proc}. This would
mean that the LPVM procedure takes in an \textit{io} resource, works on it,
modifies it, and returns it. In a later pass, these common extra resource
parameters could be optimised to point to just one memory location.

While passes of the compiler during transformation can be discussed
independently, they are not truly sequential. Some passes and transformations
happen at the same time and are dependent on each other. Nevertheless, we can
talk about the important transformations that occur on the Wybe source code as
its LPVM form is built.

\subsection{Flattening Pass}

During compilation every top level structure is converted to a procedure quite
early in the pipeline. The functions and expressions are \textit{normalised} to
look like a procedure definition along with the flattening step by the
compiler. The output of the function is simply added as a out flowing parameter
in its procedure form. A procedure body contains a sequence of statements which
build up the outputs. Hence all expressions forms have to be flattened to
appear as statements too. Some common conversions are shown in
Figure~\ref{fig:wybe_convert_to_proc}. The top level statements in the module
are put into a separate nameless procedure. The code generation phase will
later make this the \textit{main} function of the module.

Since LPVM primitives are in the form of procedure calls, all normalised Wybe
statements are gradually reduced to procedure calls since primitives in LPVM
are just procedure calls. These primitive procedure calls can be \textit{local}
calls to other procedures in the module or imported Wybe modules (fully
qualified procedure names), or they can be \textit{foreign} calls. Foreign
calls reference procedures or instructions which have to be addressed later by
the code generation phase. This is usually done by linking in an external
object file or using LLVM instructions. For example, the Wybe standard library
defines a procedure \textit{println} whose body statements are foreign calls to
C's \textit{printf}. A shared C library will be linked with the standard
library to resolve these calls to access system IO. To Wybe and LPVM the only
difference between a local and a foreign procedure call is that the local calls
can be in-lined since their definitions will have a LPVM form in another wybe
module. Otherwise it is just another \textit{Prim} (primitive) in a LPVM clause
body.

The result of the flattening pass is a very normalised structure of only
procedures. Apart from the above conversions, the normalisation which happens
along with flattening also sets up the \textit{Module} type which will hold the
internal representation of a Wybe module in the compiler. Sub-module
definitions and type definitions which occur in the same source module are
separated into separate \textit{Module} types and their module names are
qualified with the parent module name.

Normalisation will also expose the defined types and public procedure names in
the module, and add default implementations of assignment and comparison
procedures for user defined types. Note that the $=$ symbol is used for denoting
assignment and comparison here. They are differently evaluated depending on the
modes and number of parameters. here A user introduced type constructor is
normalised to a new procedure. For example, the \textit{ADT} definition for
\textit{bool} has two constructors, \textit{true} and \textit{false}, which
create the following procedures:
\begin{align*}
  type\ bool\ = public\ false\ \vert true
  \quad \rightarrow & \quad
                      \mathbf{false}(?t:bool) \\
                    & \quad
                      \mathbf{true}(?t:bool) \\
                    & \quad
                      \mathbf{=}(?out:bool, in:bool) \\
                    & \quad
                      \mathbf{=}(out:bool, ?in:bool) \\
                    & \quad
                      \mathbf{=}(x:bool, y:bool, ?z:bool) \\
\end{align*}

Normalisation in a compiler ensures that we will be operating with a well
defined and closed set of primitives hereafter. Addition of new syntax features
can be independently done as long as they can be normalised to the same set of
primitives. This greatly simplifies compiler construction.

\subsection{Type Checking Pass}

Wybe, and LPVM, is statically typed. The type checking pass is essential to
enforce this strictness and robustness of the type system. It also connects
polymorphic procedure calls to the correct procedures by matching call
signatures with procedure prototypes. Explicit typing is also only enforced on
Wybe function and procedure prototypes. The variables in the body will have
inferred types based on the input and output types.

Every variable name in the AST will be annotated with an inferred type at the
end of this pass. As mentioned above, every type in Wybe is defined in its own
module which contains all the procedures used to interact with that
type. Hence, this pass also connects type names to the modules that provide the
definition for that type. This is even required for standard types like
\textit{int} since can be provided by a non standard library just as
easily.

Polymorphic calls are resolved by looking into imported modules in reverse for
a matching procedure prototype. Since every module auto imports the
\textit{wybe} module, the type modules defined in it will be in the imported
module list for any particular module. For example, the equality procedure
\textit{'='}, can be defined in a type module for \textit{int} and the type
module for \textit{string}. The type checker will choose one depending on the
context. A statement comparing two \textit{int} (inferred) variables the call
\textit{proc call =(a, b, ?c)}, will be annotated as \textit{proc call
  wybe.int.=(a:wybe.int, b:wybe.int, ?c:wybe.int)}.

At the end of a successful type checking pass, every flattened procedure call
and variable types names will have an annotation of the fully qualified module
that defines it.


\subsection{Un-branching Pass}

The un-branching pass is where all conditional branches and loops are replaced
with procedure calls and recursion respectively. This is the structure defined
by LPVM. At this stage, a flattened Wybe procedure may create one or more
generated procedures to act as branching blocks, as described in the chapter on
LPVM. 


\subsection{Clause Generation Pass}

By this pass the LPVM structure is more or less completed. The Wybe source
semantics have been encapsulated in the LPVM IR, at least what is needed. The
implementations for all the flattened clauses of the procedures will have the
constructor \textit{ProcDefPrim} applied, with the LPVM prototype and body
primitives put in its place. Procedure calls reflect the clause they are trying
to actually call with a suffixed index. For example a call to \textit{foo<2>}
is referencing its third clause in order. 

After this pass an entire module is just a list of clauses. This representation
will be passed through further optimisation passes and finally LLVM code
generation. This clausal form is also independent of the Wybe source. Given
this form directly, we can still optimise it and generate the same code with it
as if we were given the Wybe form. This is an inherent property of an
Intermediate Representation and that's why it is so useful to have in compiler
construction.


\subsection{Expansion Pass}

An early optimisation that makes use of the LPVM structure is the in-lining of
calls. This pass takes place on the clausal or LPVM form (\textit{ProcDefPrim}
constructor in the implementation). LPVM makes heavy use of procedure calls
since every form of branching is essentially replaced with procedure
calls. In-lining bodies of \textit{callee} procedures replaces most of these
calls and avoids procedure call overheads. The decision to inline is based on a
cost vs benefit analysis. This heuristic takes into consideration the number of
callers to that procedure, it's valid arguments, and body size in number of
statements.

In-lining benefits are only decidable if we have the LPVM clause form for that
procedure. The structure of LPVM procedures is much more flat and limited as
compared to Wybe source code, making copying procedure bodies easier, and the
decision to do so even more fine grained. For a procedure body to replace its
call, the variables in the body have to be renamed while pasting. This renaming
process is simpler with LPVM procedures since they list their input and output
variables in the signature. It is only these variables which will have to be
renamed to fit in the procedure body it is being pasted into. This form of
In-lining is supported by having LPVM forms of all modules involved in
compilation in the compiler pipeline, and will be reflected in the incremental
features later. 

As an example, most of Wybe standard library types like \textit{int}
,\textit{bool}, \textit{float}, have simple primitive procedures for binary
operations. The implementation bodies for these procedures are just a single
foreign instruction call to LLVM instructions. These will definitely be
in-lined and in code generation phase will turn into a single LLVM instruction.




%%% Local Variables:
%%% mode: latex
%%% TeX-master: "../thesis"
%%% End:
