\chapter{Abstract}

Compiler efficiency in an optimising compiler can be increased in a variety of
ways. The usual optimisation methods target the Intermediate Representation
(IR) and the code generation phases to achieve better memory and time
performance. The compilation time for a build is also dependent on the number
of passes the compiler will make over its internal representations, and the
number of source modules it has to compile. This thesis explores different ways
to reduce build times by making the build process more incremental and
work-saving so that a compiler can obtain a net gain in compilation
turn-around-time by avoiding redundant work over a sequence of builds. These
features are a part of the compiler and build system \textit{Wybemk}, for a new
programming language called \textit{Wybe}. A significant portion of the
incremental features is derived from the use of a new logic language based
Intermediate Representation, called \textit{LPVM}, in the compilation
pipeline. Modelling the build system after the GNU Make utility, the
\textit{Wybemk} compiler utilises the structure of an object file to embed meta
data into it, which is used in subsequent compilations to reload redundant
work. \textit{Wybemk} includes an \textit{LLVM} back-end to provide code
generation.

%%% Local Variables:
%%% mode: latex
%%% TeX-master: "../thesis"
%%% End:
