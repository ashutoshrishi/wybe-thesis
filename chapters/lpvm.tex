\chapter{Implementation of LPVM}
\label{chap:lpvm}

The actual implementation of the LPVM structure differs a bit from the abstract
structure proposed in \cite{lpvm2015}. When a guard goal is encountered in a
clause body, the abstract representation rules result in another clause being
created with the same sequence of goals up until that guard goal. In the
implementation however, to avoid duplication, there is only one clause
generated. Assuming that a basic clause body is the sequence of goals barring a
guard goal, then at the guard goal a fork is created which contains the fork
condition, and a list of subsequent basic clause bodies. Each of the basic
clause body in the list corresponds to a different evaluation of the fork
condition (Figure~\ref{fig:forking}). Thus, a binary condition will have two
basic bodies in the fork list. This is similar to basic block branching in SSA,
however the diverging branches don't need to converge into another block and
hence there will be no need for \textit{phi} like functions.

The Figure~\ref{fig:lpvm_data_type} shows the algebraic data type used to hold
the LPVM IR in the compiler implementation. This representation has Wybe
sensitive information like Wybe types and module fields stripped away for
easier discussion. In the implemented data type, the term \textit{Goal} is
replaced with \textit{Prim}, for primitive. These \textit{primitive} statements
are meant to reflect source and target code semantics. They are just procedure
calls and the only semantic information they contain is the procedure name and
the signature. These can be used to refer to source language procedures or
machine code (or LLVM) instructions just as easily. Local calls look like:
\textit{factorial(tmp\$10: int, ?tmp\$3: int)}, and foreign calls look like:
\textit{foreign llvm mul(tmp\$2: int, 9: int, ?tmp\$3: int)}, where the term
\textit{llvm} specifies the group which can possibly provide an instruction for
the call. The compiler will decide what code to generate for a given
\textit{Prim}.

In a way, LPVM procedures or predicates are polymorphic since a call to them
will execute any one of the clauses under them. There can be a \textit{Clause}
created for a different combination of modes of the procedure parameters. For
example, the procedure \( - \) can have clauses \( -(a, b, ?c) \) and
\( -(a, ?b, c) \), marking different operand positions as outputs. Even though
each of the clause functions in a single mode, a procedure can be made to
exhibit multiple modes of a logic programming language. This construct is very
similar to the polymorphism Wybe has to offer, and as such this construct
directly enables it it.


\begin{figure}
  \centering
  \begin{tabular}{l@{\hskip 1in} | l}
    \textit{pseudo code} & \textit{lpvm} \\ \hline
    \( x0 = x1 + x3 \)              & \( wybe.int.+(x1, x3, ?x0) \) \\
    \( if\ (x0 > 0)\ left\ right \) & \( wybe.int.>(x0,\ 0,\ ?tmp1) \) \\
                                    & \( case\ tmp1\ of \) \\
                                    & \( 0:\ left.. \) \\
                                    & \( 1:\ right.. \) \\
    

  \end{tabular}
  \label{fig:forking}
  \caption{Forking in the \textit{Clause} on a \textit{Guard} conditional}
\end{figure}

\begin{figure}
  \centering
  \begin{tabular}{r c l}
    \( Proc \)   & \( \rightarrow \) & \( Clause* \) \\
    \( Clause \) & \( \rightarrow \) & \( Proto\ Body \) \\
    \( Proto \)  & \( \rightarrow \) & \( Name\ Param* \) \\
    \( Param \)  & \( \rightarrow \) & \( Name\ Type\ Flow \) \\
    \( Body \)   & \( \rightarrow \) & \( Prim*\ Fork \) \\
    \( Prim \)   & \( \rightarrow \) & \( PrimCall\ |\ PrimForeign \) \\
    \( Fork \)   & \( \rightarrow \) & \( Var\ Body* \) \\
                 & \( \rvert \)      & \( NoFork \) \\
  \end{tabular}
  \caption{LPVM Implementation Data Type}
  \label{fig:lpvm_data_type}
\end{figure}


\noindent\begin{figure}
  \vskip 1em
  \noindent\begin{minipage}{.4\textwidth}
    \begin{Verbatim}[frame=single,framesep=2mm,commandchars=\\\{\}]
\textbf{do} a
   \textbf{if} b: Break
   \textbf{else}: c
\textbf{end}
d
    \end{Verbatim}
  \end{minipage}
  \noindent\begin{minipage}{.5\textwidth}
    \begin{tabular}{r c l}
     $ \mathbf{proc}\ next1\ $ & $ \rightarrow $ & $ a\ gen1\ \mathbf{end} $\\
     $ \mathbf{proc}\ gen1\  $ & $ \rightarrow $ & $ \mathbf{guard}\ b\ 1:
                                                   break1\ \mathbf{end} $ \\
                               & $ \rvert $      & $ \mathbf{guard}\ b\ 0:
                                                   gen2\ \mathbf{end} $ \\
     $ \mathbf{proc}\ gen2\ $  & $ \rightarrow $ & $ c\ next1\ \mathbf{end} $ \\
     $ \mathbf{proc}\ break1\ $ & $\rightarrow$ & $ d\ \mathbf{end} $ \\
    \end{tabular}
  \end{minipage}
  \caption{LPVM Procedures generated for an imperative loop}
  \label{fig:loop_lpvm}
\end{figure}


Conditional statements or goals partition a clause body as discussed above. But
a loop conditional can't just do that as it would require a jump back into a
previous body at the end of an iteration. LPVM does not have these, and in fact
these are part of the SSA drawbacks it wants to avoid. Loops are un-branched by
generating new procedures which are involved in recursive calls. The calls are
tail calls and hence the code generation will have to ensure that the tail call
optimisation is enabled. There are two new procedures generated in the
un-branching: The first will contain the loop body primitives and will end up
calling itself for iteration. The second will contain the body primitives which
comes after the loop. The clause body that the loop was a statement in
originally will end with a call to the first generated procedure. The first
procedure, in turn, will contain a breaking conditional or guard to call the
second generated procedure.

If we consider \textit{a}, \textit{b}, \textit{c}, \textit{d} as a
representation for one or more body statements or primitives, a looping
construct looks like \( \mathbf{do}\ a\ b\ \mathbf{end}\ c\ d\). The two
generated procedures for it will can be
\( \mathbf{next1:}\ a\ b\ next1\ \mathbf{end} \) and
\( \mathbf{break1:}\ c\ d\ \mathbf{end} \). A more compound example with
conditional breaking is shown in Figure~\ref{fig:loop_lpvm}. The guard goal(s)
\textit{b} decides whether to exit the loop or not, and \textit{c} represents
the remaining body of the loop after the condition. The conditional inside the
loop split the generated procedure \textit{next1} into \textit{gen1} and
\textit{gen2}. As shown above, in the implementation the two clauses of
\textit{gen2} will be present in a \textit{Fork} based on the value of
\textit{b}. The original loop statement will now just be a call to
\textit{next1}. Since LPVM procedures support multiple output, we can have the
same scoped variables which came out of the loop block as the outputs in
\textit{next1}.


\begin{figure}
  \noindent
  \begin{minipage}{0.2\textwidth}
    \begin{align*}
      gcd(a,b,?ret) &\rightarrow \mathbf{guard}\ b \neq 0 \\
                    &\wedge mod(a,b,?b') \\
                    &\wedge gcd(b,b',?ret) \\
      \\
      gcd(a,b,?ret) &\rightarrow \mathbf{guard}\ b == 0 \\
                    &\wedge ret=a \\
    \end{align*}
  \end{minipage}
  \qquad\vrule
  \begin{minipage}{0.4\textwidth}
        \begin{align*}
      \mathbf{gc}&\mathbf{d}(a,b,?ret) \rightarrow \\
          & foreign\ \mathbf{llvm}\ \mathrm{icmp}\ \mathrm{ne}(b,0,?tmp1) \\
          & case\ \mathrm{tmp1}\ of \\
      0:\ & foreign\ \mathbf{llvm}\ \mathrm{move}(0,?ret) \\
      1:\ & foreign\ \mathbf{llvm}\ \mathrm{urem}(a,b,?b') \\
          &  gcd(a,b',?ret) \\
    \end{align*}

    \end{minipage}

    \caption{Comparing the clause form of GCD (left) with its actual LPVM
      implementation (right)}
    \label{fig:gcd_full_imp}
\end{figure}


%Conclusion paragraph?
Figure~\ref{fig:gcd_full_imp} shows the implemented LPVM version of the
\textit{gcd} function we saw earlier. On the left is the abstract form
representation, and on the right is how it looks like in the actual
implementation. The optimisation and analysis methods given in \cite{lpvm2015}
are still possible for the above implemented structure. The LPVM structure also
does not define a fixed instruction set, in fact its instruction set is being
provided by LLVM. LPVM exists as an important glue structure between Wybe and
machine instructions in the form of LLVM.


%%% Local Variables:
%%% mode: latex
%%% TeX-master: "../thesis"
%%% End:
