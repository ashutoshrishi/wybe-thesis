\chapter{LPVM}

\section{Horn Clauses as an Intermediate Representation}

The paper describes a new form of IR using a logical programming structure, now
called LPVM. The wybe compiler presents an implementation of this clausal form
and as such LPVM is a core part of its pipeline. This thesis provides the final
code generation stage for LPVM and wybe.

LPVM is an attempt to make program analysis and transformation easier. It's an
alternative to the common imperative IR forms while maintaining a similar set
of program analysis methods. The paper compares LPVM against the other current
options for an IR, like the three address code and SSA, discussing the
drawbacks of these common IRs. The focus is on solving these drawbacks at a
more fundamental level rather than relying on further transformations. The
motivation outlined in the paper is having a simple limited structure in
analysis, while drawing in existing logical programming reasoning. The limited
set of primitives, additionally enables the wybe compiler to be more efficient
by embedding a LPVM transformation in the object file. This would have been
inefficient with a more complex structure.

The Three Address Code (TAC) IR and its refinement, the Static Single
Assignment (SSA) form, are popular IRs used in compiler constructions. They are
simple enough to be universal and are quite extensible to accommodate different
source language semantics due to their little restrictions. A SSA based IR will
generally be laid out as basic blocks and branching instructions connecting
them, like a graph. The SSA refinement requires all variable names to be unique
in a block. This makes it easier to track variable lifelines. But
this also requires a virtual function called phi to choose the value of the
same variable coming in from two different blocks. The paper points out that
this is inadequate for determining the control flow path. While the phi
function can tell us which blocks are providing the values for the same
variable during analysis, it won't tell us the condition that was evaluated for
that decision. This can be done by looking into all the predecessor blocks that
provided the value. Another solution to this, as pointed out by the paper, is
using the Gated Single-Assignment (GSA) form which augments the existing phi
function to capture the block entering condition.

Another drawback described is the SSA form being biased to forward
analysis. The phi function at the top of a block is helpful for analysing that
block locally. A predecessor block (to the block with phi) just knows about its
branching destination, and not about the phi function choosing between its
variables and other blocks. A transformation to the standard SSA form, called
the Static Single Information (SSI), provides a solution to this problem by
adding another virtual function to the end of the blocks which describes the
variables that will be compare with other blocks in the subsequent block.

So far every drawback of SSA has been addressed can be solved by creating an
extension to the SSA structure. While these are perfectly feasible, they are
additional complexities. In the case of LPVM these problems are solved at the
basic structural level, without any special functions. During analysis, the
special functions of SSA have to be resolved with extra work. An analysis of a
block containing phi node involves revisiting the analysis work done for the
predecessor blocks. This is a result of SSA's naming rules. Even
though SSA enforces unique names for every declaration, alternate blocks can
still have the same variable names. A more functional translation of SSA
avoids this, but still exhibits the forward bias problem.

% Don't just summarise the drawbacks, don't need to, just say its there and
% discussed, LPVM does better at a structural level. And its important to
% understand these common drawbacks since SSA is pop-u-lar.

% Explanation of the LP form
The LPVM IR is a restricted form of a logical language. It does not feature
non-determinism seen commonly in a LP. Therefore all input parameters have to
be bound to a value before calling that procedure. It also requires fixing the
mode of a parameter. In LP, a procedure can have a parameter which can behave
as an input or an output. LPVM requires this behaviour to the explicit and
fixed for every parameter for easier reasoning. These restrictions makes LPVM
surprisingly easy to read.

% Present a simple explanation of horn clauses
At the top level LPVM form there are only procedures. In the form presented in
the paper a procedure consists multiple \textit{Horn Clauses}. A horn clause
with its \textit{Head} describing its parameters and procedure name, and the
sequence of goals as the body can be seen as a procedure itself. Since LPVM
wants to be deterministic, only one of the clauses with the same name evaluated
for given inputs. That clause can be seen as the entire procedure itself with
the \textit{Head} as the procedure signature, without bothering about the usual
backtracking in LP. The parameters to a clause also needs its modes to be
explicitly defined: either as an input or an output. There may be two
alternative \textit{Clauses} of the same name having switched up modes for the
same parameters. Since determinism will select only one of them as the
procedure, single modedness is preserved.

The \textit{Goals} are simple body statements. Logical statements in SSA result
in block cause branching. These statements are represented as guard goals in
LPVM. When a guard goal is encountered, a new clause will be created with the
same goal sequence until that guard, but will follow the complimentary
evaluation of the guard after that. In the implementation however, the
procedure body or the clause body or the sequence of goals are put in a tree
data structure. Instead of having somewhat duplicated sequence of goals in
creating an alternate clause, the guard goals instead create branching child
nodes, each with further sequence of goals. This is similar to the layout of
basic blocks within a function in SSA, which makes it easier to translate the
LPVM to any SSA based IR later for code generation. 

At the end of conversion to LPVM, the IR form only contains procedures with
multiple clauses. Alternate branches of conditionals are incorporated in their
own clauses, chosen by the evaluation of a guard statement. Unconditional jumps
are procedural calls and loops are flattened to recursive procedure
calls. Every procedure will also define all its input and output parameters
explicitly in its signature. Since output variables are specified here, there
is no need of a return instruction. The only variables that would exist in the
clause bodies are now the specified input and output variables, and temporaries
between them. This declarative paradigm greatly simplifies a lot of
analyses. This also helps determine the purity of a procedure. 


\section{Current Implementation of LPVM}


\begin{figure}
\centering
\begin{tabular}{l@{\hskip 1in} l}
  \( x0 = x1 + x3 \)              & \( wybe.int.+(x1, x3, ?x0) \) \\
  \( if\ (x0 > 0)\ left\ right \) & \( wybe.int.>(x0,\ 0,\ ?tmp1) \) \\
                                  & \( case\ tmp1\ of \) \\
                                  & \( 0:\ left.. \) \\
                                  & \( 1:\ right.. \) \\
                                

\end{tabular}
\\
\caption{SSA statement and their equivalent LPVM goals}
\end{figure}

\begin{figure}
\centering
\begin{tabular}{r c l}
\( Proc \)   & \( \rightarrow \) & \( Clause* \) \\
\( Clause \) & \( \rightarrow \) & \( Proto\ Body \) \\
\( Proto \)  & \( \rightarrow \) & \( Name\ Param* \) \\
\( Param \)  & \( \rightarrow \) & \( Name\ Type\ Flow \) \\
\( Body \)   & \( \rightarrow \) & \( Prim*\ Fork \) \\
\( Prim \)   & \( \rightarrow \) & \( PrimCall\ |\ PrimForeign \) \\
\( Fork \)   & \( \rightarrow \) & \( Var\ Body* \) \\
\end{tabular}
\caption{LPVM Implementation Data Type}
\label{fig:lpvm_data_type}
\end{figure}

The Figure~\ref{fig:lpvm_data_type} shows the algebraic data type used to hold
the LPVM IR. This representation has wybe sensitive information like wybe types
and module definitions stripped away for easier discussion. The actual
implementation will be looked at again in a latter chapter.


In the LPVM implementation used by the Wybe compiler every goal is a primitive
procedure call. At the most basic level these are similar to primitive machine
instructions. This again simplifies code generation. As mentioned above,
instead of having duplication of clause bodies the guard goals currently result
in a fork of bodies. Generation of two complementary clauses similar to the
LPVM specifications in the paper is trivial from this tree structure.


All loops and conditionals are un-branched by generating new procedures which
are involved in recursion. The calls are tail calls and hence the code
generation will have to ensure tail call optimisation is enabled. Simple
conditionals like if conditions, just generate a fork in the body. The
alternate destinations, say \textit{Body A} and \textit{Body B}, are stored in a
list in the \textit{Fork}. Since the \textit{Fork} is at the end of a clause
\textit{Body}, the remaining body (after the If statement) is placed in a new
generated procedure. This generated procedure is called at the end of
\textit{Body A} and \textit{Body B}.

% Explain Loops 
