\chapter{Literature Review}


\section{Horn Clauses as an Intermediate Representation}

The paper describes a new form of IR using a logical programming structure, now
called LPVM. The wybe compiler presents an implementation of this clausal form
and as such LPVM is a core part of its pipeline. This thesis provides the final
code generation stage for LPVM and wybe.

LPVM is an attempt to make program analysis and transformation easier. It's an
alternative to the common imperative IR forms while maintaining a similar set
of program analysis methods. The paper compares LPVM against the other current
options for an IR, like the three address code and SSA, discussing the
drawbacks of these common IRs. The focus is on solving these drawbacks at a
more fundamental level rather than relying on further transformations. The
motivation outlined in the paper is having a simple limited structure in
analysis, while drawing in existing logical programming reasoning. The limited
set of primitives, additionally enables the wybe compiler to be more efficient
by embedding a LPVM transformation in the object file. This would have been
inefficient with a more complex structure.

The Three Address Code (TAC) IR and its refinement, the Static Single
Assignment (SSA) form, are popular IRs used in compiler constructions. They are
simple enough to be universal and are quite extensible to accommodate different
source language semantics due to their little restrictions. A SSA based IR will
generally be laid out as basic blocks and branching instructions connecting
them, like a graph. SSA requires all variable names to be unique in a
block. This restriction makes it easier to track variable lifelines. But this
also requires a virtual function called phi to choose the value of the same
variable coming in from two different blocks. The paper points out that this is
inadequate for determining the control flow path. While the phi function can
tell us which blocks are providing the values for the same variable during
analysis, it won't tell us the condition that was evaluated for that
decision. This can be done by looking into all the predecessor blocks that
provided the value. Another solution to this, as pointed out by the paper, is
using the Gated Single-Assignment (GSA) form which augments the existing phi
function to capture the block entering condition.

The next issue the paper describes is the SSA form being biased to forward
analysis. The phi function at the top of a block is helpful for analysing that
block locally. A predecessor block (to the block with phi) just knows about its
branching destination, and not about the phi function choosing between its
variables and other blocks. Another transformation to the standard SSA form,
called the Static Single Information (SSI), as pointed to by the paper, solves
this problem. It adds another virtual function to the end of the blocks to fill
this information gap. 

So far every drawback of SSA has been addressed by other projects by creating
an extension to the SSA structure. While all of these are perfectly feasible,
they are additional complexities. In the case of LPVM these problems are solved
at the base structural level, without any special functions. It identifies the
drawback of SSA to lie in its naming rules. 












