\chapter{Wybe Build System}

Wybemk is the wybe compiler and wybe make system joined together in one
utility. The wybemk compiler just needs the name of a target to build, and it
will decide the building and linking order. Targets include architecture object
files, LLVM bitcode files, or a final linked executable. 

Similar to the gnu make utility, it does not want to rebuild any object file
which is already built. Taking it one step further, the object files wybemk
makes store useful LPVM information generated from the source code. While a
normal object file does present it's name table, allowing the functions stored
in it to be called as extern functions, these functions are not
inlineable. While source code for functions can be stored in a section of the
object file, it is easier to just pass the original source file around with the
object file. The LPVM structure of a parsed source file however is simpler in
its structure. A body of a function is simply a list of clauses. Storing this
information in the object file enables inlining in at the LPVM generation
stage. Inlining also presents a wider view for analysis as we can look at the
body of the functions we want to call externally. 

Storing a serialised form of LPVM in the object file is quite similar to the
LLVM bitcode files, which stores the LLVM IR in a byte file. To keep this
option open, since we use LLVM as the final stage, we also present a wrapped
LLVM bitcode structure which stores both the LLVM and LPVM IR in it. Such a
file can be linked in by wybemk or by the LLVM compiler, which loads different
information from the file.


\section{Storing structures in Object files}

Object files store relocatable object code which is the compiled code generated
by the LLVM compiler in the wybe compiler. Even though different architectures
have their own specification of the object file format, they are modelled
around the same basic structure. Object files defines segments, which are
mapped as memory segments during linking and loading. The segment TEXT usually
contains the instructions. An object file also lists the symbols defined in it,
which is useful for the linker to resolve external function calls from another
module being linked. Avoiding all the common segment names, it is possible to
add new segments to the object file (at the correct bytes), which do not get
mapped to memory. Using such such segment we can attempt storage of some useful
serialised meta-data in the object file.

To emulate the behaviour of gnu make, or to avoid rebuilding an object file, it
is quite trivial to check the file creation date and reach a decision. However
this hinders a lot of LPVM level optimisation. Since there is no need to
compile a source module if it has a newer object file for it, we can't do any
inter-module analysis with it. We can get the symbol names and possibly the
generated object code for the function names, but those are not useful for LPVM
analyses. Storing generated LPVM code in the object files would circumvent
this. The simple semantics of LPVM is an excellent candidate for storage in
object files without being disruptive. Even when the source code file is
present, loading the LPVM code from the object file is faster than engaging the
parser and generating LPVM from the parse tree.

LLVM also generates bitcode files. These files store a serialised form of the
generated LLVM IR. The LLVM compiler can transform these to the machine local
object file too. Being a simple bitstream format, it is possible to include
other serialised metadata into this file as long as the LLVM compiler can find
its required IR. Storing LPVM in these files is a feasible alternative to the
machine local object file formats. LLVM bitcode file can be used for code
generation on any valid target, much like the Java bytecode. 

\section{LPVM as a stored structure}

The simplicity of LPVM makes it effecient...


\section{Loading Inlienable LPVM}

Talk about object file file structures?

