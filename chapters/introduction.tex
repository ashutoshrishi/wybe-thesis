\chapter{Introduction}

This thesis presents an incremental and work-saving compiler which utilises a
newly proposed logic programming based intermediate representation, called
\textbf{LPVM}, in its pipeline. The source language for this compiler is a new
multi-paradigm language called \textbf{Wybe} (pronounced \textit{Wee-buh}). The
compiler eventually targets the \textbf{LLVM} framework which allows it to be
flexible in targeting multiple modern architectures and actually generating
machine code. Ideas explored in this thesis stem from the motivation of
achieving efficiency in a compilation process by avoiding any work that has
been done in an earlier compilation. The term \textit{work-saving} here, means
that the compiler should be able to recognise scenarios where it can reload
completed previous work rather than do the same steps again. For any particular
build it should recompile only those source files which really need to be
recompiled. Being \textit{incremental} means being able to define a complete
compilation as a function of smaller individual compilations, and then working
on only those smaller units which have changed. The units which have not
changed need not be recompiled. The level of this smaller compilation unit is
the \textit{granularity} of a compiler. Traditionally, a compiler operates at
the granularity of a source file. We want to work with something smaller than
that, without interrupting normal operations of a compiler.

The purpose of the Wybe programming language is to be an easy to learn language
without sacrificing robust static type systems which is often sought after in a
real world project. Wybe wants to serve the same domain as languages like
Python, in being easy to read and reason with, and statically typed languages
like Java, in being suitable for production. It also brings with it the power
of declarative languages and efficient compiled code. A work-saving compiler
for Wybe, which is geared to eliminate redundancy, should further push its
suitability for projects involving a large number of modules. With LLVM as the
back-end code generator we can produce more efficient machine code as the
project moves along to maturity.

The choice of an intermediate representation to use in the compiler is an
important one. It is the data structure which holds an abstract representation
of the source code semantics, while being machine independent. It undergoes
multiple optimisation passes which may or may not alter its structure, and will
ultimately be targeted to generate code for any given architecture. The choice
of the IR structure and form makes certain program analyses easier to do than
the rest. A particular form may even enable some specific transformation
technique which other forms might not inherently allow. In a compilation
pipeline, there can multiple IR forms between the source code and the machine
code \citep{irnote}. IRs which resemble the source code semantics more closely
are deemed to be on a higher level than the ones which have a closer
resemblance (and the restrictions) to some machine code or assembly. An IR will
strip away information it does not need anymore, or information that it has
utilised completely, and transform to a flatter lower level form as it moves
through a compilation process. In our pipeline we utilise two different forms
of IR. The \textbf{LPVM IR} form is generated from a source program in Wybe,
and the \textbf{LLVM IR} form is generated from the final LPVM IR form. All the
major optimisation and analysis passes happen on the LPVM form. The LLVM form
is generated on a simplified version of LPVM (after optimisation passes) and is
mostly needed so that we can generate machine code without duplicating the
effort in the numerous code generators the LLVM project already includes.

% talk about LPVM
LPVM IR is the implementation of the logic programming based IR form given in
\cite{lpvm2015}. The term \textit{LPVM} is an abbreviation for \textit{Logic
  Program Virtual Machine}, and named in a similar style to \textit{LLVM}. They
are not really virtual machines, but as an IR, they provide an abstract
instruction set resembling an abstract machine. While the use of LPVM
inherently provides plenty of benefits in program analysis and optimisation, we
can also exploit its structure to achieve the incremental features in our
compiler. LPVM has an incredibly simple IR structure which makes doing program
analysis and reasoning easier than its counterparts. It also allows the use of
powerful logic programming analyses. In this thesis though, the focus is more
on exploring the structural benefits of LPVM in providing incremental features
in the compiler, rather than its proposed logic programming optimisation
techniques.

The Wybe compiler is called \textbf{Wybemk} (pronounced
\textit{Wee-buh-mik}). It is a combination of the words \textit{Wybe} and
\textit{Make}. To eliminate redundant recompilation, we need some internal
compiler structures passed on from a previous compilation to be compared with
the current compilation. By identifying parts which can be re-used in the
current compilation process we can skip a lot of passes. To facilitate this
forwarding of information we make use of the byte structure of a object
file. Since building relocatable object files is the natural end goal of any
compiler, we don't have to develop new Wybemk specific container
formats. Hence, we explore ways of embedding information into architecture
dependent object files. These object files will appear as ordinary object files
to every other compiler, but will contain extra docile information usable by
Wybemk.

Another focus of the Wybemk tool is to have a build system ingrained into the
compiler. We try to mimic the GNU \textit{Make} utility \citep{make} with some
simplifications. To behave like Make, we recognise object files which are newer
than the corresponding source file and avoid re-building that object file. But
we also don't want to depend on external Makefiles and interface/header
files. Wybemk is intelligent enough to determine the transitive closure of
dependencies to build for any give source. The goal here is to provide a
complete solution for a build system so that there is no need to rely upon
third party tools and build systems later when the language matures.


The rest of the thesis presents a more in-depth discussion of the points made
above, and the work done to build this incremental compilation pipeline for
Wybe. The sections try to follow the stages of the compilation in order. In
Chapter~\ref{chap:literature_review} we discuss the literature explored to
build the compiler, including discussions on the important papers which form a
base for this thesis. Especially the paper introducing LPVM. The Wybe
programming language is then introduced in Chapter~\ref{chap:wybe}. Having an
overview of the actual language being compiled makes discussion easier. In
Chapter~\ref{chap:lpvm} we present the actual implementation of LPVM used in
the compiler, and in Chapter~\ref{chap:wybe_to_lpvm} we discuss the
transformation of Wybe to LPVM. This is the first stage of the compiler. We are
then set to present the incremental and work saving build system, Wybemk, in
Chapter~\ref{chap:build_system}. Finally we discuss code generation from LPVM
to LLVM in Chapter~\ref{chap:codegen_llvm}. The thesis is tangential to a
working implementation of the Wybemk compiler, which is written entirely in
Haskell.


%%% Local Variables:
%%% mode: latex
%%% TeX-master: "../thesis"
%%% End:
