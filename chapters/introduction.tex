\chapter{Introduction}

This thesis presents an incremental and work-saving compiler which utilises a
newly proposed logic programming based intermediate representation, called
\textbf{LPVM}, in its pipeline. The compiler itself compiles a new
multi-paradigm source language called \textbf{Wybe} (pronounced
\textit{Wee-buh}). The compiler eventually targets the \textbf{LLVM} compiler
framework which allows the compiler to be flexible in targeting multiple modern
targets, and provide a working implementation. Ideas explored in this thesis
are stemmed from the motive of achieving efficiency in a compilation pipeline
by avoiding redoing any work done before. The source language Wybe is a
complete full featured language targeted at being easy to learn and being
suitable for production use. While the use LPVM itself provides us plenty of
benefits in program analysis and optimisation, we also exploit its structure to
achieve incremental features in our compiler. This will push Wybe to be even
more suitable for large scale projects.

The choice of the intermediate representation to use in the compiler is an
important one. It is the data structure which abstractly holds the source code
semantics, while being machine independent. It undergoes many optimisation
passes which may or may not alter its structure, and can be targeted to any
architecture for code generation eventually. The choice of the IR structure
also makes certain program analyses easier than the rest, or even enables some
which other structures might not allow. There can multiple IR forms between the
source code and the machine code \citep{irnote}. IRs which resemble source code
semantics more closely are deemed to be on a higher level than the ones which
have a closer resemblance to the machine code. An IR will strip away
information it does not need anymore, or information that it has utilised
completely, and transform to a simpler lower level form. In our pipeline we
have two IRs. The \textbf{LPVM IR} form is generated from the source program in
Wybe, and the \textbf{LLVM IR} form is generated from the final LPVM IR
form. All the major optimisation and analysis passes happen on the LPVM
form. The LLVM form is generated on a simplified version of LPVM (after
optimisation) and is just needed so that we can generate machine code without
duplicating effort.

% talk about LPVM
LPVM IR is the implementation of the logic programming IR form given in
\cite{lpvm2015}. The term \textit{LPVM} is an abbreviation for \textit{Logic
  Program Virtual Machine}, and named so in a similar fashion to
\textit{LLVM}. They are not really virtual machines. LPVM has an incredibly
simple IR structure and makes it really easy to for reasoning with different
program analysis techniques. It also allows the use of powerful logic
programming analysis. In this thesis, the focus is more on exploring the
structural benefits of LPVM in being incremental, rather than its proposed and
possible optimisation techniques.

The Wybe language compiler is called \textbf{Wybemk} (pronounced
\textit{Wee-buh-mik}). It is a combination of \textit{Wybe} and
\textit{Make}. To support our incremental features, we need to embed certain
information on a compilation into the object files the compiler creates. These
object files are the same as the architecture specific object files, but with
extra docile information usable by Wybemk. Another focus of the Wybe compiler
is to have a build system ingrained into the compiler. We try to mimic the gnu
\textit{Make} utility with some simplifications.

% Write outline of the thesis

The rest of the thesis presents the research and work done to build the
incremental compilation pipeline for Wybe. The sections try to follow the stages of
the compilation in order. In Chapter~\ref{chap:literature_review} we discuss
the literature explored to build the compiler, including discussion on the
important papers which form a base for this thesis. In Chapter~\ref{chap:lpvm}
we present the implementation of LPVM used in the compiler. In
Chapter~\ref{chap:wybe}, an introduction to the Wybe programming language is
given, and in Chapter~\ref{chap:wybe_to_lpvm} its transformation to LPVM is
discussed. Having described the structure of the source language, LPVM, and its
representation in the compiler, we present our incremental and work saving
build system, Wybemk, in Chapter~\ref{chap:build_system}. Finally we discuss
code generation to LPVM in Chapter~\ref{chap:codegen_llvm}.



%%% Local Variables:
%%% mode: latex
%%% TeX-master: "../thesis"
%%% End:
