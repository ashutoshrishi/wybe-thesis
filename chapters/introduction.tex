\chapter{Introduction}


This thesis explores using a logical intermediate representation (IR) in a
incremental compiler pipeline for a new declarative and imperative mixed
paradigm source language. This logical IR is called LPVM, and its simple
clausal structure enables logical programming analysis and easier
optimisations. It lies middle of the compilation pipeline, before an LLVM
generation stage, making it a middle level IR. Using this representation, we
can also make the compiler incremental and have a lazy build system which tries
to avoid re-compilation as much as it can. Even though having multiple IR
stages and forms adds extra work in the compilation process (and compiler
construction), we show a simple transformation of LPVM to LLVM, so that the
compiler is able to target all the machines LLVM can realistically.

The source language is called Wybe. It's a new language which aims to unify
the good parts of declarative and imperative languages. Having a mixture of
paradigms makes it a good fit for a compilation pipeline which involves a
mixture of paradigms as well. 

The data structure of an Intermediate representation is an important factor in
deciding what optimisation passes are going to be useful. Different compilers
usually have their own IR, which maybe only slightly different from other
IRs. The actual form is really a compiler construction choice. There are
efforts to build a universal IR, like the LLVM project, but a reasonably
complex language would have it's own unique requirements which can't be
accounted for in a single universal IR or form. The IR generation stage in a
compiler pipeline goes through multiple optimisation passes. There is no
restriction on the form of the IR as it moves through these passes. In fact
having multiple IR forms, which gradually transform from being closer to the
source language to a more machine dependent form is quite common. Multiple
forms opens up multiple approaches to optimisations. 

Mid level IRs, quit simply lie in between some high level IR or source code and
a low level IR. We insert LPVM into the compilation pipeline as a mid level
IR between the source code and the LLVM IR. In a way, our target code is the
LLVM IR. Even though LLVM code generation comes with it's own set of curated
and tested optimisations, its main use here is to avoid the need to account for
every architecture. Thus we can focus on maximising the usefulness of
LPVM. We also show how the clausal form of LPVM can be easily translated to the
more imperative block style of LLVM.

The Wybe compiler wants to be lazy and incremental. It wants to avoid as much
recompilation as it can. The object files Wybe compiler builds have enough
information to be used in place of a source file, while at the same time
provide inalienable code which could only have been obtained from the source
code. The simple clausal semantics of LPVM is the key to this. To be more
incremental, the compiler maps and tracks the source code semantic structures
to LPVM clauses. For re-compilation, it tries to load the already compiled
clauses stored in the object files for any trivial changes of the source. 

Another focus of the Wybe compiler is to have a build system ingrained into the
compiler. We try to mimic the gnu make utility with some simplifications, and
hence our Wybe compiler is usually run through the command `wybemk` (Wybe
make). Instead of having a separate make file, the Wybemk command just takes a
target name, infers the dependency chains and the list of files to compile and
link and makes the target. We wanted to have the ability to link in foreign
source object files and not just Wybe source files. 



Why use a logical IR?
