\chapter{Future Work}
\label{chap:future_work}

% Wybe
The Wybe programming language still has a lot of syntax changes and new
features coming along in its future pipeline. It's a new language and complete
syntax re-works are common. The changes to Wybe syntax naturally changes a lot
of existing LPVM structures, which in turn have to be dealt with in LLVM code
generation. Work on adding more LPVM analyses and optimisation passes is also
underway. We also want to add a more complete static garbage collection option
to Wybe and extend the typing implementation to be more space effecient. All
modern languages feature memory robustness and safety, and Wybe will have to
compete with them

The build system of the Wybemk compiler is consistently being made more
incremental. Some of the unimplemented possibilities have been discussed in
Chapter~\ref{chap:build_system}. The hash function used for comparing
\textit{checksums} can also be chosen more intelligently.

Along with newer approaches to be more time-saving, we also want to extend our
compiler to be more cross platform. We have shown how we embed information into
\macho object files for OS X only. Using the cross platform LLVM
\textit{bitcode} files is still an option. Currently Wybemk can build an object
archive target for a folder of Wybe source files. This is the first step
towards packages in the build system. However we do not embed any information
into the archive files yet. Our use of normal object files makes it possible to
link Wybe with object files built by other compilers. Since Wybemk is supposed
to be a complete solution for compiling and building, it should be intelligent
enough to know which system libraries to link in for any use-case. For example,
Wybe should be able to explicitly say that it wants to be linked with some C
GUI library on the system.

While we want to exploit the new logic IR, LPVM, as much as possible and use
LLVM for code generation, we don't want to duplicate LLVM optimisation
methods. The LLVM back-end framework provides numerous optimisation passes for
which a lot of work has already been put it.  We don't want to shy away from
these. Testing the effects and incorporating more advanced LLVM optimisation in
tandem with LPVM passes are planned for future work.


%%% Local Variables:
%%% mode: latex
%%% TeX-master: "../thesis"
%%% End:
