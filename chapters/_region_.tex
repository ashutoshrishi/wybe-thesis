\message{ !name(wybe.tex)}
\message{ !name(wybe.tex) !offset(-2) }
\chapter{Wybe programming Language}


\begin{figure}
\centering
\begin{tabular}{r c l}

\( func\ factorial(n:int):int \) & \(\rightarrow \) & 
\( proc\ factorial(n:wybe.int, ?\verb+$+:wybe.int) \) \\
\( ?c = bar(a, b) \) & \(\rightarrow\) & \( bar(a, b, ?c) \) \\
\( ?y = f(g(x)) \) & \(\rightarrow\) & \( g(x, ?temp)\ f(temp, ?y) \) \\


\end{tabular}
\\
\caption{Normalisation of Wybe functions to Procedures.}
\label{fig:wybe_convert_to_proc}
\end{figure}


Wybe is a new programming language that exhibits both declarative and
imperative paradigms. It wants to incorporate all the good ideas of both and
still be simple to learn. Wybe is statically typed with a strong preference for
interface integrity. Interface integrity involves depicting all the inputs and
outputs, along with side-effects in the function or procedure prototype. By
forcing the information flow to be explicit, Wybe makes it easy to determine
the purity of the function just through a function or procedure prototype. The
Wybe as a language is still under construction, with new good ideas being
tested every now and then.

With Wybe, a module is equivalent to a wybe source file. The module name is
same as the source file without the extension. The module's interface consists
of the public functions and procedures in the module. There is also a separate
syntax to declare sub-modules inside the full file module. Sub-module names are
qualified with the outer modules' name. For example a module \textit{A.B.C} is
the sub-module of \textit{A.B}, which is a sub-module of \textit{A}, which is
the module of the source file \textit{A.wybe}. Defining new types also creates
a new sub-module.


Procedures in Wybe can be polymorphic. Multiple Wybe procedures can have the
same name but with different paramter types and modes. This enables procedures
with multiple modes, similar to Prolog. For example, a procedure to add two
numbers can have the following definitions: \textit{add(x, y, ?z)},
\textit{add(x, ?y, z)}, \textit{add(?x, y, z)}. The call \textit{add(3, ?t, 5)}
will evaluate \textit{t} to be \textit{2}. This selection is done in the Type
checking pass, which matches calls to the definition with the correct types and
modes.




\section{Compilation of Wybe}

The Wybe syntax tree is slowly transformed to the LPVM IR structure. In this
process it undergoes flattening, type checking, un-branching, and a final
clause generation pass to obtain a structure similar to
Figure~\ref{fig:lpvm_data_type}. Variables in Wybe can be adorned to explicitly
define their direction of information flow. A variable can flow in (x), flow
out (?x), or both ways (!x). The Wybe model of explicit information flow is
quite similar to the LPVM predicates, making LPVM a good fit for this language,
or the opposite. During compilation everything is converted to a procedure
quite early in the pipeline. The functions and expressions are normalised to
look like a procedure definition along with the flattening step by the
compiler. The output of the function is simply added as a out flowing parameter
in its procedure form. Expressions are dealt with in a similar way.  Some
common conversions are shown in Figure~\ref{fig:wybe_convert_to_proc}.

Since LPVM primitives are just procedure calls, all Wybe statements are
gradually reduced to procedure calls too. These primitive procedure calls can
be calls to other procedures in the module or imported modules (fully qualified
procedure names), or be foreign calls. The foreign calls are used to indicate
linking of a foreign object file and its code, for example, the standard
library makes foreign calls to C, to access system IO. Even though these are
standard procedure calls to LPVM, it will later become an extern call in LLVM
to a shared object file. The only difference between a local and a foreign
procedure call is that the local calls can be in-lined. Otherwise it is just
another \textit{Prim} in a clause body.

The un-branching pass is when LPVM replaces the remaining conditionals and
loops with recursion, as shown in the last chapter. 



\message{ !name(wybe.tex) !offset(-87) }
