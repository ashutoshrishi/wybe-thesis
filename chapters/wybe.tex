\chapter{Wybe programming Language}


\begin{figure}
\centering
\begin{tabular}{r c l}

\( func\ factorial(n:int):int \) & \(\rightarrow \) & 
\( proc\ factorial(n:wybe.int, ?\verb+$+:wybe.int) \) \\
\( ?c = bar(a, b) \) & \(\rightarrow\) & \( bar(a, b, ?c) \) \\
\( ?y = f(g(x)) \) & \(\rightarrow\) & \( g(x, ?temp)\ f(temp, ?y) \) \\


\end{tabular}
\\
\caption{Normalisation of Wybe functions to Procedures.}
\label{fig:wybe_convert_to_proc}
\end{figure}


\begin{figure}


\textit{public type} \textbf{int} \textit{is} \textbf{i64} \\

\begin{itemize}[label={}]

\item
\textit{public func} 
\textbf{+}(x:int,y:int):int = foreign llvm add(x,y)

\item  
\textit{public proc} 
\textbf{+}(?x:int,y:int,z:int) ?x = foreign llvm sub(z,y) 
\textit{end}

\item 
\textit{public proc} 
\textbf{+}(x:int,?y:int,z:int) ?y = foreign llvm sub(z,x) 
\textit{end}

\item 
\textit{public func} 
\textbf{=}(x:int, y:int):bool = foreign llvm icmp eq(x,y)
\end{itemize}

\textit{end} \\

\caption{Sample of the wybe.int module from Wybe standard library}
\label{fig:wybe_int}
\end{figure}


Wybe is a new multi-paradigm programming language, featuring both imperative
and declarative constructs. At the top level it contains both functions and
procedures. A function header specifies the inputs and their types, and the
output expression type. The body of the function will be an expression
evaluating to a value of that specified out type. Whereas a procedure header
species its inputs and outputs (along with their types), and any mutable or
external resource it works with (like IO). Its body will contain sequential
statements which build those outputs from the inputs. There is no return
statement, as at the end of the procedure body the specified outputs will be
returned. In a way, a procedure header lists the parameters which will be used
in its body, and fixes the flow mode (input flow or output flow) for each of
them.

Wybe is statically typed with a strong preference for interface integrity, for
which a function or procedure header should define all it's input and output
types, along with any mutable resources that will be affected. By forcing the
information flow to be explicit, Wybe makes it easy to determine the purity of
the function just by looking at a function or procedure prototype. There is
also no requirement for an interface or header file. 

With Wybe, a module is equivalent to a wybe source file. The module name is
same as the source file without the extension. The module's interface consists
of the public functions and procedures in the module. There is also a separate
syntax to declare sub-modules inside the full file module. Sub-module names are
qualified with the outer modules' name. For example a module \textit{A.B.C} is
the sub-module of \textit{A.B}, which is a sub-module of \textit{A}, which is
the module of the source file \textit{A.wybe}. Defining new types also creates
a new sub-module. All dependencies of a module can be inferred through the top
level \textit{use module} statements in the source file.

Types in Wybe are simply modules. Standard Wybe considers {int},
\textit{float}, \textit{char}, \textit{string} as primitive types. In reality
these are just types provided by the Wybe standard library, which can be
replaced with any other flavour of a standard library. Defining basic types
requires just specifying its memory layout (in terms of word sizes) and
providing procedures or functions to interact with the basic type. A sample
from the \textit{wybe.int} type module is shown in Figure~\ref{fig:wybe_int}. This
type definition is in the source file \textit{wybe.wybe}, making \textit{int} a sub-module
of the module \textit{wybe}. 

Wybe also supports \textit{algebraic data types}.


Procedures (and even functions) in Wybe can be polymorphic. Multiple Wybe
procedures can have the same name but with different paramter types and
modes. For example, a procedure to add two numbers can have the following
prototypes: \textit{add(x, y, ?z)}, \textit{add(x, ?y, z)}, \textit{add(?x, y,
  z)}. The call \textit{add(3, ?t, 5)} will evaluate \textit{t} to be
\textit{2}. This selection is done in the Type checking pass during
compilation, which matches calls to the definition with the correct types and
modes.



\section{Compilation of Wybe to LPVM}

The Wybe syntax tree is slowly transformed to the LPVM IR structure. In this
process it undergoes \textit{flattening}, \textit{type checking},
\textit{un-branching}, and a final clause generation pass to obtain a structure
similar to Figure~\ref{fig:lpvm_data_type}. Variables in Wybe can be adorned to
explicitly define their direction of information flow. A variable can flow in
(x), flow out (?x), or both ways (!x). The Wybe model of explicit information
flow is quite similar to the LPVM predicates, making LPVM a good fit for this
language. 

\subsection{Flattening Pass}

During compilation everything is converted to a procedure quite early
in the pipeline. The functions and expressions are normalised to look like a
procedure definition along with the flattening step by the compiler. The output
of the function is simply added as a out flowing parameter in its procedure
form. Expressions are dealt with in a similar way.  Some common conversions are
shown in Figure~\ref{fig:wybe_convert_to_proc}.

Since LPVM primitives are in the form of procedure calls, all normalised Wybe
statements are gradually reduced to procedure calls too. These primitive
procedure calls can be calls to other procedures in the module or imported
modules (fully qualified procedure names), or be foreign calls. Foreign calls
reference procedures or instructions which have to be addressed later by
linking in some library which provides it. For example, the wybe standard
library defines \textit{println} whose body statements are foreign calls to C's
\textit{printf}. A shared C library will be linked with the standard library to
resolve these calls to access system IO later. To Wybe and LPVM the only
difference between a local and a foreign procedure call is that the local calls
can be in-lined since their definitions will have a LPVM form in another wybe
module. Otherwise it is just another \textit{Prim} (primitive) in a LPVM clause
body.

\subsection{Type Checking Pass}

Wybe is statically typed, so having a type checking pass is essential. Every
variable name in the AST will be annotated with an inferred type. This pass
connects type names to the modules that provide the definition for that
type. This is required as even standard types like \textit{int} can be provided
by a non standard library just as easily. Polymorphic calls are resolved to the
actual definitions here.

Type definitions include functions and procedures which work with the defined
type. For example, the equality function \textit{'='}, can be defined in a type
module for \textit{int} and the type module for \textit{string}. The type
checker will choose one depending on the context. A statement comparing two
\textit{int} (inferred) variables the call \textit{proc call =(a, b, ?c)}, will
be converted to \textit{proc call wybe.int.=(a:wybe.int, b:wybe.int,
  ?c:wybe.int)}. 


\subsection{Un-branching Pass}

The un-branching pass is when LPVM replaces the remaining conditionals and
loops with recursion, as shown in the last chapter. 


