\chapter{Wybe Programming Language}
\label{chap:wybe}

% Figure for showing the INT module
\begin{figure}

\textit{public type} \textbf{int} \textit{is} \textbf{i64}

\begin{itemize}[label={}]
\setlength\itemsep{0em}
\item
\textit{public func} 
\textbf{+}(x: int, y: int) : int = foreign llvm add(x,y)

\item  
\textit{public proc} 
\textbf{+}(?x: int, y: int, z: int) ?x = foreign llvm sub(z,y) 
\textit{end}

\item 
\textit{public proc} 
\textbf{+}(x: int, ?y: int, z: int) ?y = foreign llvm sub(z,x) 
\textit{end}

\item 
\textit{public func} 
\textbf{=}(x: int, y: int) : bool = foreign llvm icmp eq(x,y)
\end{itemize}

\textit{end} \\

\caption{Sample of the \textit{wybe.int} module from Wybe standard library}
\label{fig:wybe_int}
\end{figure}
%%%%%%%%%%%%%%%%%%%%%%%%%%%%%%%%%%%%%%%%%%%%%%%%%%



% Figure for showing the bool type
\begin{figure}
\vskip 2em
\textit{public type} \textbf{bool} = 
\textit{public} \textbf{false} \verb!|! \textbf{true}
\begin{itemize}[label={}]
\setlength\itemsep{0em}

    
\item
\textit{public func} \textbf{=}(x: bool, y: bool) : bool = 
\textit{foreign} \textit{llvm} icmp eq(x,y)

\item
\textit{public func} \textbf{/=}(x: bool, y: bool) : bool = 
\textit{foreign} \textit{llvm} icmp ne(x,y)
\end{itemize}
\textit{end}

\caption{Bool type as an Algebraic Data Type from the Wybe standard library}
\label{fig:wybe_bool}
\end{figure}
%%%%%%%%%%%%%%%%%%%%%%%%%%%%%%%%%%%%%%%%%%%%%%%%%%

Wybe is a new multi-paradigm programming language, featuring both imperative
and declarative constructs. At the top level it contains both functions and
procedures. A function header specifies its inputs and their types, and the
output expression type. The body of the function will be an expression
evaluating to a value of that specified out type, similar to a function in
other declarative languages. Whereas, a procedure header specifies its inputs
and outputs (along with their types), and any mutable or external resource it
works with (like IO). Its body will contain sequential statements which build
those outputs from the inputs. There is no return statement, as at the end of
the procedure body the specified outputs will be returned. The procedure header
lists all the important parameters which will be used in its body, and fixes
the flow mode (input flow or output flow) for each of them.

Wybe is statically typed with a strong preference for \textbf{interface
  integrity}. Both functions and procedures are able to present a pure
interface. A function or procedure header should define all of it's input and
output types, along with any mutable resources that will be affected by it. By
also forcing the information flow to be explicit, Wybe makes it easy to
determine the effect of a function or procedure just by looking at its
prototype.  Variables in Wybe can be adorned to explicitly define their
direction of information flow. A variable can flow in (x), flow out (?x), or
both ways (!x). Thus, a procedure header which looks like: $proc\ mul(a,?b,c)$,
says that the procedure \textit{mul} operates on two input parameters at
positions $a$, $c$, and builds one output at parameter position $b$. There is
also no requirement for an interface or header file. With procedures having a
preceding visibility label (\textit{private} or \textit{public}), it is very
easy to automatically generate an interface for a module.

A Wybe \textbf{module} is equivalent to a Wybe source file with the extension
\textit{.wybe}. The module name is the same as the source file name, but
without the extension. A module interface consists of the public functions and
procedures defined in it. There is also a separate syntax to declare
sub-modules inside a module. Sub-module names are qualified with the
outer/parent modules' name. For example a module \textit{A.B.C} is the
sub-module of \textit{A.B}, which is a sub-module of \textit{A}, which is the
module of the entire source file \textit{A.wybe}. This also means that
importing module \textit{A} will give you access to its sub-modules. Defining
new types also creates a new sub-module named after that type. All dependencies
of a module can be inferred through the top level \textit{use module}
statements in the source file. Sub-modules have the dependency of the
super-module automatically added to their interfaces. Hence, a sub-module can
access the interface of its sibling modules. When \textit{module A uses module
  B}, \textit{module A} can call all the public functions and procedures of
\textit{module B}.

A Wybe module can contain statements outside any procedure, as top level
statements. These are always executed in order when the module is called or
imported. The idea here is to insert module initialisation and setup code here
which is required to run even on imports. For example, a module defining an
interface to access a database can have automatic initialisation code in top
level statements to set up the connection to the database.


\begin{figure}

\begin{Verbatim}[frame=single,label=Simple Statement,labelposition=topline,framesep=4mm,fontsize=\small,commandchars=\\\{\}]
println("Hello, World!")
\end{Verbatim}

  \begin{Verbatim}[frame=single,label=Factorial Function,labelposition=topline,framesep=4mm,fontsize=\small,commandchars=\\\{\}]
public func factorial(n:int):int =
    if n <= 0 then 1 else n * factorial(n-1)
\end{Verbatim}

    \begin{Verbatim}[frame=single,label=Factorial Procedure with IO,labelposition=topline,framesep=4mm,fontsize=\small,commandchars=\\\{\}]
public proc factorial(n:int) use !io
    ?result = 1
    do while n > 1
       ?result = result * n
       ?n = n - 1
    end
    println(result)
end
\end{Verbatim}

    \begin{Verbatim}[frame=single,label=User Types,labelposition=topline,framesep=4mm,fontsize=\small,commandchars=\\\{\}]
public type position = public position(x:int, y:int) end

\end{Verbatim}
  
  
  \caption{Sample Wybe programs}
  \label{fig:wybe_programs}
\end{figure}

\textbf{Types} in Wybe are simply modules. Standard Wybe considers {int},
\textit{float}, \textit{char}, \textit{string} as primitive types. In reality
these are just types provided by the Wybe standard library module, which can be
replaced with any other flavour of a standard library. Defining basic types
just requires specifying its memory layout (in terms of word sizes) and
providing procedures or functions to interact with the basic type. A sample
from the \textit{wybe.int} type module is shown in
Figure~\ref{fig:wybe_int}. This type definition is in the source file
\textit{wybe.wybe}, making \textit{int} a sub-module of the module
\textit{wybe}. This \textit{int} type sets the size of its members to be
\textit{i64}, a syntax borrowed from LLVM, occupying 64 bits.

More compound types can be defined in terms of basic types or other compound
types. Wybe has \textit{Algebraic Data Types} too. The \textit{bool} type can
be defined as an ADT with two type constructors, as shown in
Figure~\ref{fig:wybe_bool}. The compiler will infer the storage for members of
this type. 

Procedures (and even functions) in Wybe can be \textbf{polymorphic}. Multiple
Wybe procedures can have the same name but with different paramter types and
modes. For example, a procedure to add two numbers can have the following
overloaded prototypes: \textit{add(x, y, ?z)}, \textit{add(x, ?y, z)},
\textit{add(?x, y, z)}. The call \textit{add(3, ?t, 5)} will evaluate
\textit{t} to be \textit{2}. This selection is done in the Type checking pass
during compilation, which matches calls to the definition with the correct
types and modes. The above definitions model multiple parameter modes of a
logic language in a way. Having the three definitions as above, the procedure
\textit{add} can be called with any single of it's parameters un-bound (as an
output).

\textbf{Resources} in Wybe are a used to specify some data that the procedure
uses or modifies without explicitly being a part of the procedure
arguments. This is an easy and pure way to list the side-effects of a
procedure. An example of a resource in standard Wybe is \textit{IO}. When a
procedure says it is using the \textit{IO} resource, we know it will be
accessing some IO interface along with building its output parameters. For
example: \textit{public proc greet use io}. Custom resources can also be
defined.



On the imperative side, Wybe does have loops. Figure~\ref{fig:wybe_programs}
shows a few examples of Wybe programs to introduce the general syntax. The
procedure form of the factorial operation reveals that it will be doing IO in
its body using the \textit{use !io} syntax. 


%%% Local Variables:
%%% mode: latex
%%% TeX-master: "../thesis"
%%% End:


