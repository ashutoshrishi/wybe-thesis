\chapter{Code Generation to LLVM}


We do not generate machine code directly using LPVM IR. Instead we are hooking
into the LLVM project so that we are able to compile Wybe programs on multiple
architectures without duplicated effort. Without the LLVM we would have to
write code generators for every popular architecture and more. LLVM is a
popular open source project. A lot of older compilers (and new) are being
re-targeted to LLVM IR, and the efforts taken previously in making the machine
code generator efficient is being poured collectively into LLVM.

The LLVM IR has an abstract structure of an imperative program, and is a SSA
based form. Earlier we talked about how LPVM IR solves the drawbacks of the
using SSA naming scheme and virtual functions, and yet we are ultimately
targeting a SSA based IR. We do this based on the observations that we are
doing majority of our optimisation and program analysis in the LPVM stage, and
the simplified LPVM form does not need the \phif to be present in the LLVM
IR. In its final stage, LPVM procedure body primitives behave like calls to
other procedures or virtual instruction calls. Generating LLVM IR on these
bodies is very direct.

Being imperative means we have to transform each LPVM procedure into a
function. Wybe and LPVM IR support procedures with multiple outputs, but
functions don't. This flexibility should be reflected in the LLVM IR as
well. Having multiple outputs are an abstraction of placing values in multiple
registers at the end of the execution of a procedure body. In LLVM each
register is virtually reflected by a variable name or an aggregate
structure. We wrap the multiple values which are to be returned by a procedure
in a register structure. The values are unwrapped in the body of the caller in
the same order.

The virtual instruction calls in LPVM are in the form of \textit{foreign}
procedure calls. An example is \textit{foreign llvm add(a:int, b:int,
  ?c:int)}. This refers to the LLVM IR instruction \textit{add} which adds
two integers. LLVM instructions are function calls. Transforming a procedure
call to a function call is quite trivial in most scenarios. The function call
version for the above example is: \textit{c = add i32 a, b} (actual variable names
look much different in LPVM and LLVM).

LPVM replaces branching and jumps with procedure calls. These are tail
calls. For LLVM IR to generate efficient machine code, it is extremely
important to ensure that the LLVM tail call optimisation is turned on. This
requires turning on certain flags in the \textit{call} instruction generated
in LLVM IR, and ensuring the call is indeed a tail call in the imperative
function body representation of the procedure.

%%% Local Variables:
%%% mode: latex
%%% TeX-master: "../thesis"
%%% End:
