\chapter{Code Generation to LLVM}
\label{chap:codegen_llvm}

We do not generate machine code directly from the LPVM IR form. Instead we are
hooking into the LLVM framework \citep{Lattner:MSThesis02} so that we are able
to compile Wybe programs on multiple architectures without additional
effort. In this chapter this transformation from LPVM to the LLVM IR form is
presented. Without the LLVM project we would have to write code generators for
every popular architecture and more. LLVM is a popular open source project with
a lot of collective effort being poured into it. A lot of older mature
compilers (and new) are also being re-targeted to LLVM IR, and the efforts
taken previously in making the machine code generator efficient is being poured
into LLVM. We feel it's worthwhile to do the same for Wybe.

The LLVM IR has an abstract structure of an imperative program. It is a SSA
based form. Earlier we talked about how LPVM IR solves the drawbacks of the SSA
naming scheme and its virtual functions, and yet we are ultimately targeting a
SSA based IR. We do this based on the observation that we are doing majority of
our optimisation and program analysis in the LPVM stage, and the simplified
LPVM form does not need the \phif to be present in its LLVM IR
transformation. In its final stage, basic LPVM \textit{clause} bodies end with
either building it's outputs or a fork. This fork does not merge back into
another block. Instead, every other basic body in the fork will either
terminate finishing its work or will end in a call to another procedure. With
no converging bodies, there is no need of using a \phif in LLVM. Generating
LLVM IR on these bodies is very direct. We take a deeper look at these
conversions in section~\ref{sec:lpvm_to_llvm}.

Using the compiler tools provided by LLVM framework, we can create object code
from the LLVM IR easily. We then use the system linker to link these object
files. For accessing instructions not provided by LLVM, we link in a custom
shared C library. Currently this library allows access to a few standard IO
functions and the \textit{stdlib} of C. Since the C compiler \textit{clang}
compiles C using LLVM, we can either join the LLVM form of this library with
ours or just using the system linker to link object file versions. Using this
library we can provide a stronger support for compound types which need access
to heap memory, as discussed in section~\ref{sec:type_codegen}. Finally we
discuss how Wybemk does its dependency linking and deals with top level code
for every dependency in section~\ref{sec:linking}.


\section{Transforming LPVM to LLVM}
\label{sec:lpvm_to_llvm}

Since LLVM is based on an imperative form, we have to transform each LPVM
procedure back to an imperative function. An imperative function and a
procedure are somewhat similar in that they both have a sequence of imperative
statements as their body. Wybe and LPVM IR support procedures with multiple
outputs, but imperative functions do not do so by default. We need the
flexibility of multiple outputs reflected in the LLVM IR as well. Having
multiple outputs is an abstraction of placing values in multiple registers at
the end of the execution of a procedure body. In LLVM each register is
virtually reflected by a variable name or an aggregate structure. We wrap the
multiple values which are to be returned by a procedure in a register
\textit{structure}. The values are unwrapped in the body of the caller in the
same order. This aggregate structure type is allocated on the stack of the
function instead of the heap, making it similar to a local variable. The
nameless procedure which holds the top level statements of the module is
transformed to a \textit{module.main} function. The prefixed module name is
used to differentiate from \textit{main} functions of other modules.

% describe the LLVM form here
The LLVM IR looks like a C program. At the top level, it contains
\textit{declare} statements (to refer to an extern function or some global
constant) and \textit{functions}. A function has a return type and parameters
defined exactly like a C function. Void functions are acceptable. The body of a
function is composed of basic blocks. A basic block ends with a
\textit{terminator} statement which is either a return statement
(\textit{ret}) or a branching statement (\textit{br}). A basic block will branch
into alternate basic blocks. Eventually the basic blocks converge, possibly to
a block with a $\varphi$\textit{-function}. Statements in a basic block are
\textit{instruction} calls. LLVM provides numerous predefined
\textit{instructions} for which it can generate machine code. They model
\textit{instructions} of an assembly language. Examples of some
\textit{instructions} are: \textit{add}, \textit{sub}, \textit{icmp},
\textit{store} .etc. LLVM uses pointers and indexing to access memory locations
in the heap. Allocation of heap memory is left up to the programmer to
implement. The correctness of the bounds in indexing is also left to the
programmer. There are no exceptions. LLVM provides a few compound data
structures like the structure, array, and vector as well. Being a high level IR
it assumes an abstraction of infinite registers. The LLVM framework also
provides numerous tools and utilities to compile, assemble, disassemble, link,
optimise, .etc. LLVM IR modules.

Referring to the structure in Figure~\ref{fig:lpvm_data_type}, we know that a
LPVM primitive statement is either a \textit{local} procedure call or a
\textit{foreign} call. We mentioned earlier that these calls are meant as a
directive for the compiler to generate specific machine code. For a
\textit{foreign} call of the form: \textit{foreign group proc\textunderscore
  name(..args..)}, we expect an \textit{instruction} provided by the given
\textit{group} to replace the call in the generated LLVM IR. Currently in our
implementation, we have three types of foreign groups: \textit{c},
\textit{llvm}, and \textit{lpvm}. Foreign calls with the group \textit{c} refer
to functions defined in the shared C library file which is linked in later. In
LLVM IR these are called by first declaring this function name as an
\textit{extern} and then generating a \textit{call} instruction for it. Foreign
calls with the group \textit{llvm} refer to LLVM instructions directly by
name. The \textit{args} will be type tested and the validity of the call
ensured at code generation. Foreign calls to \textit{lpvm} are special
functions whose implementation can be provided by anyone. Currently we have
memory allocation, access, mutation functions in this group, and we use
\textit{C} functions to provide the implementation. This is discussed in more
detail in section~\ref{sec:type_codegen}.

The \textit{GCD} function for which we showed the LPVM transformation earlier,
is now shown being translated to the LLVM IR in
Figure~\ref{fig:lpvm_to_llvm}. The two clause branches spawned off the
condition on goal \textit{b} end in a \textit{return} statement and a procedure
call respectively. An \textit{foreign} call like \textit{foreign llvm
  add(a:int, b:int, ?c:int)} will have to converted to a function call. This
call translates to the LLVM IR instruction \textit{add} which adds two
integers. Transforming a procedure call to a function call is quite direct in
most scenarios. The function call version for the above example is: \textit{c =
  add i32 a, b} (actual variable names look much different in LPVM and LLVM).

We have seen that LPVM replaces loops with tail call recursion. For LLVM IR to
generate efficient machine code, it is extremely important to ensure that the
LLVM tail call optimisation is turned on. This requires mentioning certain
flags in the generated \textit{call} instruction and ensuring that the correct
calling convention is used. We also have to tell the LLVM code generator to
optimise these marked calls with the \textit{-tailcallopt} optimisation
flag. Finally, the function call must be an actual tail call. Since the block
has to return a value in its terminating statement, a function call can't
actually be in the tail position of a function. However, if a return
instruction follows the actual call and returns the value returned by the
function or returns void, that call is considered to be a tail call. In
Figure~\ref{fig:lpvm_to_llvm} the recursive call to \textit{gcd} is an example
of a LLVM tail call.



\begin{figure}
    \begin{align*}
      \mathbf{gc}\mathbf{d}(a,b,?ret) & \rightarrow \\
          & foreign\ \mathbf{llvm}\ \mathrm{icmp}\ \mathrm{ne}(b,0,?tmp1) \\
          & case\ \mathrm{tmp1}\ of \\
      0:\ & foreign\ \mathbf{llvm}\ \mathrm{move}(0,?ret) \\
      1:\ & foreign\ \mathbf{llvm}\ \mathrm{urem}(a,b,?b') \\
          &  gcd(a,b',?ret) \\
    \end{align*}
\begin{Verbatim}[frame=lines,label=LLVM,labelposition=topline,framesep=4mm,fontsize=\small,commandchars=\\\{\}]
define i64 @gcd(i64 %a, i64 %b) \{
  %0 = icmp ne i1 %b, 0 
  br i1 %0, label %if.then, label %if.else
if.then:
  %1 = urem i64 %a, %b
  %2 = tail call i64 @gcd(i64 %a, i64 %1)
  ret %2
if.else:
  ret i64 0
\}
\end{Verbatim}
    \caption{Comparison of the LPVM form of \textit{GCD} to its LLVM form}
  \label{fig:lpvm_to_llvm}
\end{figure}


The LLVM target independent code generator provides numerous optimisation
passes. While generating code, these passes can be specified by their name and
LLVM will run them in order before generating assembly code. A curated set of
passes selected by the LLVM team is also provided under a single compound
pass. All of these passes can be run during code generation or on an LLVM IR
\textit{bitcode} file using the \textit{llvm-opt} utility. Currently in our
implementation we are using an elementary set of passes so that we can focus on
LPVM more. For every new pass added to the Wybemk pass set, we have to ensure
that it is not redoing the analysis LPVM has already done. Similarly, for every
pass LPVM wants to implement, we check if LLVM already provides some equivalent
pass.


\section{Code generation for Wybe types with LLVM}
\label{sec:type_codegen}

LLVM provides access to a basic integer type of arbitrary bit precision and the
float type. The syntax \textit{i32} represents an integer with 32 bit precision
in LLVM. In Wybe, we are considering an \textit{int} to be of the machine word
size, which is commonly \textit{i32} or \textit{i64}. However, the bit size of
the standard \textit{int} type can be specified through Wybe type syntax as
shown in Chapter~\ref{chap:wybe}. Using the basic LLVM types we can model all
of Wybe's basic types like \textit{int}, \textit{float}, \textit{bool},
\textit{char} .etc.

Compound types and \textit{Abstract Data Types} require allocation on the heap.
For algebraic data types, the compiler will determine the byte sizes (or word
sizes) for every type constructor used and will generate \textit{access} and
\textit{mutate} procedure calls accordingly. As mentioned in
Chapter~\ref{chap:wybe_to_lpvm}, a Wybe type definition results in the
generation of \textit{setter} and \textit{getter} procedures for members of
that type. One such generation is shown in Figure~\ref{fig:procs_adt} for an
ADT \textit{position} which represents a geometric Cartesian
coordinate. Internally, for LLVM, the type \textit{position} will be
represented as a pointer to a word size integer (i64* or i32*). A
\textit{position} type pointer will point to the location where its members are
located on the heap. The \textit{position} type constructor takes two integers
to build itself. Wybemk determines that this needs a total of 16 bytes
allocated on the heap, 8 for each member (assuming an int is defined as
\textit{i64} here). In the two clauses shown for the constructor procedure for
\textit{position}, clause 0 packs up two integers into a \textit{position} type
and clause 1 unpacks two integers from a \textit{position} type. The
\textit{mutate} and \textit{access} procedure calls pass a constant
\textit{int} as the offset index.

On the side of LLVM, the instructions \textit{getelementptr}, \textit{store},
\textit{load} are used to index a pointer, store data using that index, and
load data from that index respectively. The \textit{foreign} calls shown above
all translate to some combination of these three LLVM instructions, along with
\textit{casting} instructions. In a way, \textit{alloc}, \textit{mutate},
\textit{access} are LPVM instructions, with their machine code provided in LLVM
assembly. 


\begin{figure}
  \ruleline{Wybe}
  \begin{minipage}{\textwidth}
    \begin{equation*}
      public\ type\ \mathbf{position} = public\ \mathbf{position}(x:int, y:int) end
    \end{equation*}\\
  \end{minipage}
  \ruleline{LPVM}
  \begin{minipage}{\textwidth}
    \begin{align*}
      0: \mathbf{posi}&\mathbf{tion}(x:int, y:int, ?out:position): \\
          & foreign\ \mathrm{lpvm}\ \mathbf{alloc}(16:int, ?temp\#0:position) \\
          & foreign\ \mathrm{lpvm}\ \mathbf{mutate}(temp\#0:position, ?temp\#1:position,
            0:int, x:int) \\
          & foreign\ \mathrm{lpvm}\ \mathbf{mutate}(temp\#1:position, ?temp\#2:position,
            1:int, y:int) \\
          & foreign\ \mathrm{llvm}\ \mathbf{move}(temp\#2:position,
            ?out:position) \\ \\
      1: \mathbf{posi}&\mathbf{tion}(?x:int, ?y:int, out:position): \\
                      & foreign\ \mathrm{lpvm}\ \mathbf{access}(out:position,
                        0:int, ?x:int) \\
                      & foreign\ \mathrm{lpvm}\ \mathbf{access}(out:position,
                        1:int, ?y:int) \\
    \end{align*}
  \end{minipage}
  
  \label{fig:procs_adt}
  \caption{LPVM Procedures generated for an Abstract Data Type}  
\end{figure}

By default, LLVM only provides heap indexing instructions. The allocation has
to be implemented by the programmer. An elementary way to do this is to use
LLVM function calls to C \textit{stdlib}, for which an object file providing
access to this library has to be linked in later. By using calls to
\textit{malloc} and \textit{free}, we already have a simple memory interface
set up. These calls return a \textit{void} pointer which can be
\textit{bitcasted} to the required pointer type using LLVM
\textit{instructions}. The LLVM instruction \textit{getelementptr} can then
index the heap allocated memory. With this solution, the garbage collection, or
calling \textit{free} at the correct places, becomes a big task. Since we also
want to make headway into providing garbage collection in the compiler, it
would be better to have a more intelligent memory interface.

Without providing a complete solution, we can still access dynamic garbage
collection by making use of the The Boehm-Demers-Weiser conservative garbage
collector \citep{boehmgc}. Accessing the C library version of this through the
shared library we link in later, we can use the replacement function for
\textit{malloc}, called \textit{gc\textunderscore malloc}, to enable the Boehm
GC to automatically collect memory. We don't have to worry about making correct
calls to \textit{free} anymore. To setup these calls, LPVM generates
\textit{foreign} calls in the \textit{lpvm} group: \textit{foreign lpvm
  alloc(8:int, x:vector)}, where \textit{vector} is an example new type name.


\section{Linking Dependencies}
\label{sec:linking}

With LLVM we can link a transitive closure of a modules' dependencies in two
ways: build object files for each module and use the system linker, or use the
\textit{llvm-link} utility to link LLVM IR forms of each module together into
one file and build a single object file of that. Since we are exploiting the
structure of object files to support our incremental features, we are inclined
to follow the first approach.

In the current implementation we are invoking the system \textit{ld} linker
when we have the object files ready. When the linker is given multiple object
files to link in an executable, it expects only one of the passed object files
to define the \textit{main} symbol. Currently modules with a non empty set of
top level statements have a \textit{module.main} procedure defined, which is
the entry point for that \textit{module}. This procedure should be executed
when its module is imported into another or when it is the target
executable. This procedure is transformed to a function of the same name in
LLVM IR. During linking, for every module in the dependency graph, all of these
local \textit{module.main} function bodies should be concatenated into a single
body, in depth first order of import, and made available under the
\textit{main} symbol of the executable object code. For example, let's assume
we are building an executable target $foo$ and the dependency chain looks like:

\begin{equation*}
  foo \leftarrow aaa,bbb \\
  aaa \leftarrow ccc,ddd \\
  bbb \leftarrow ddd
\end{equation*}

The \textit{module.main} functions ($aaa.main()$, $bbb.main()$...) are
concatenated in the order: $ccc,ddd,aaa,bbb,foo$. Notice how $ddd.main$ wasn't
considered twice even though it would appear twice in the traversal. There is
no need to run the initialisation code for that module again. This concatenated
body is put into a new temporary object file under the function name
\textit{main} and added to the linking list so that \textit{ld} can find the
\textit{main} symbol it needs.


As mentioned above, apart form the Wybe object files, we are also linking in a
C shared library. Since Wybemk is supposed to be a complete build system
solution for Wybe, it should also be able to link in dynamic system libraries
built with other compilers. A Wybe source module should be able to request
linking of some external object file. This is up-coming in the future
versions. Such a construct would also replace the need for the current C shared
library we use. Instead, the usage of some C header/library can be explicitly
specified, and Wybemk will search the system library paths for it. The goal is
to keep the compiler and standard library independent. This is similar to how
many languages provide their standard library separately from the compiler. Our
independence also extends to the implementation of primitive types in the
compiler, since the Wybe source file of the standard library can state the size
of \textit{ints} in bits.





%%% Local Variables:
%%% mode: latex
%%% TeX-master: "../thesis"
%%% End:
